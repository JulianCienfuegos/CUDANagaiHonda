\chapter{CHASTE}
\section{Introduction}
CHASTE is an open source computational framework for modeling epithelial tissues. This  software has forces built in to represent a number of forces, including the Honda-Nagai force, the Weliky-Oster force, as well as forces defined by the researcher Farhadifar. The program features a variety of mesh generators which simulate various boudary conditions, and can accept several cell-proliferation conditions. Although this is not of immediate concern to us, the software can also recreate several other fixed-lattice models, which models cells as existing on a fixed lattice of points. That is to say, in these models space is divided into a grid, and cells can occupy space inside the grid blocks, but cannot half fill the blocks. A cell can occupy several blocks in this model.

CHASTE also features a variety of ODE and PDE solvers, though the software uses only a simple Euler integrator to move the vertices in the vertex dynamics simulation. One obvious way to improve the software would be to include more complicated and stable solvers.

\section{Some Examples From the Tutorial}
Here is the output from the simulation "TestCelllBasedDemo" which uses a simple hexagonal mesh generator, a pseudorandom cell proliferation condition, and the Honda-Nagai force. The output from the program is displayed using a lightweight java program which the authors wrote for the purpose of their code. The output is shown in figures 1 and 2.


In this next model, a confined tissue is simulated in which  the cells are not allowed to grow outside of some elastic binding box. Once again, the output shows the tissue at the initial time, and after 20 units of time have passed. The output is shown in figrues 3 and 4.


These few examples offer only a glimpse of ease of use of the software. The code is also generously commented and there is an active user base which speaks regularly over the community mailing list.

\section{A modified example} 
In this example I generate a 2x2 mesh of bounded cells. Then, I create a force which  simulates a strong push from the left. Then, I specify a TransitCellProliferativeType whic specifies how the cells proliferate as they move. The initial and final conditions  are shown in figures 5 and 6.

\begin{lstlisting}
#include <cxxtest/TestSuite.h>
#include "CheckpointArchiveTypes.hpp"
#include "AbstractCellBasedTestSuite.hpp"

#include "AbstractForce.hpp"
#include "HoneycombMeshGenerator.hpp"
#include "FixedDurationGenerationBasedCellCycleModel.hpp"
#include "GeneralisedLinearSpringForce.hpp"
#include "OffLatticeSimulation.hpp"
#include "CellsGenerator.hpp"
#include "TransitCellProliferativeType.hpp"
#include "SmartPointers.hpp"

#include "FakePetscSetup.hpp"

class MyNewForce : public AbstractForce<2>
// This class createsa new force.
{
private:

    double mStrength;
    //The force has some strength
    friend class boost::serialization::access;
    template<class Archive>
    void serialize(Archive & archive, const unsigned int version)
    {
        archive & boost::serialization::
			base_object<AbstractForce<2> >(*this);
        archive & mStrength;
    }

public:
    MyNewForce(double strength=1.0)
        : AbstractForce<2>(),
          mStrength(strength)
    {
        assert(mStrength > 0.0);
    }

    void AddForceContribution(AbstractCellPopulation<2>& rCellPopulation)
    {
        c_vector<double, 2> force = zero_vector<double>(2);
        force(0) = mStrength;
        // The force is a 2d vector, where the 0th entry is the x
        // direction, and the 1st entry is the  y direction.

        for (unsigned node_index=0; 
			node_index<rCellPopulation.GetNumNodes(); node_index++)
        {
            rCellPopulation.GetNode(node_index)->
				AddAppliedForceContribution(force);
			// Apply the force to each node in the mesh.
        }
    }

    double GetStrength()
    {
        return mStrength;
    }

    void OutputForceParameters(out_stream& rParamsFile)
    {
        *rParamsFile << "\t\t\t<Strength>" << 
			mStrength << "</Strength>\n";
        AbstractForce<2>::OutputForceParameters(rParamsFile);
    }
};

#include "SerializationExportWrapper.hpp"
CHASTE_CLASS_EXPORT(MyNewForce)
#include "SerializationExportWrapperForCpp.hpp"
CHASTE_CLASS_EXPORT(MyNewForce)

class TestCreatingAndUsingANewForceTutorial 
	: public AbstractCellBasedTestSuite
{
public:
	
    void TestMyForce() throw(Exception)
    {
        HoneycombMeshGenerator generator(2, 2);
        MutableMesh<2,2>* p_mesh = generator.GetMesh();
		// Make a mesh
        std::vector<CellPtr> cells; //  Create  some cells in the mesh.
        CellsGenerator<FixedDurationGenerationBasedCellCycleModel, 2> 
			cells_generator; //Specify how the cells proliferate.
        cells_generator.GenerateBasic(cells, p_mesh->GetNumNodes());

        MeshBasedCellPopulation<2> cell_population(*p_mesh, cells);
        // Get the mesh ready to be modified.

        for (unsigned i=0; i<cell_population.GetNumNodes(); i++)
        {
             cell_population.GetNode(i)->ClearAppliedForce();
             // Clear the force on each node in the mesh to be safe.
        }

        MyNewForce force(1.0); // Create a new force.

        force.AddForceContribution(cell_population); // Add the force
		// To each node in the mesh

        for (unsigned node_index=0; 
			node_index<cell_population.GetNumNodes(); node_index++)
        {
            TS_ASSERT_DELTA(cell_population.GetNode(node_index)->
				rGetAppliedForce()[0], 0.0, 1e-4);
            TS_ASSERT_DELTA(cell_population.GetNode(node_index)->
				rGetAppliedForce()[1], -5.0, 1e-4);
			// Check for numerical errors in the movement of the nodes.
        }

        OutputFileHandler handler("archive", false);
        std::string archive_filename = 
			handler.GetOutputDirectoryFullPath() + "my_force.arch";
        {
            AbstractForce<2>* const p_force = new MyNewForce(2.6);
            std::ofstream ofs(archive_filename.c_str());
            boost::archive::text_oarchive output_arch(ofs);

            output_arch << p_force;
            delete p_force;
        }
        {
            std::ifstream ifs(archive_filename.c_str(), std::ios::binary);
            boost::archive::text_iarchive input_arch(ifs);

            AbstractForce<2>* p_force;
            input_arch >> p_force;

            TS_ASSERT_DELTA(dynamic_cast<MyNewForce*>(p_force)->
				GetStrength(), 2.6, 1e-4);

            delete p_force;
        }
    }
    

    void TestOffLatticeSimulationWithMyForce() throw(Exception)
    {
		// Now that we have tested the force for errors, we 
		// can use the force in our simulation.
        HoneycombMeshGenerator generator(2, 2);
        MutableMesh<2,2>* p_mesh = generator.GetMesh();
		// create a mesh
        std::vector<CellPtr> cells;
        MAKE_PTR(TransitCellProliferativeType, p_transit_type);
        CellsGenerator<FixedDurationGenerationBasedCellCycleModel, 2> 
			cells_generator;
        cells_generator.GenerateBasicRandom
			(cells, p_mesh->GetNumNodes(), p_transit_type);

        MeshBasedCellPopulation<2> cell_population(*p_mesh, cells);

        OffLatticeSimulation<2> simulator(cell_population);
        simulator.SetOutputDirectory("TestMyForce");
        simulator.SetSamplingTimestepMultiple(12);
        simulator.SetEndTime(10.0);
		// set up the simulator
        MAKE_PTR_ARGS(MyNewForce, p_force, (0.5));
        simulator.AddForce(p_force);

        MAKE_PTR(GeneralisedLinearSpringForce<2>, p_linear_force);
        p_linear_force->SetCutOffLength(1.5);
        simulator.AddForce(p_linear_force);
		// add forces
        simulator.Solve();
        // solve the simulation.
    }
};
\end{lstlisting}

