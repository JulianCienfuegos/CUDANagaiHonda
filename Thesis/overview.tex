\chapter{Overview of the Literature}
\section{Introduction}
Lately, vertex dynamics models have acquired a great deal of attention for their ability to accurately simulate the morphogenesis of epithelial tissue. The models all agree that epithelial tissues can be approximated by a mesh of polygons. Where these models distinguish themselves, however, is in how they specify the forces which act upon the vertices of the mesh. Some models, such as the Weliky-Oster model, specify explicit forces which act upon the vertices. Others, such as the Honda-Nagai model, assume that the forces acting upon the vertices can be described as the gradient of the free energy function, which wants to minimize itself.
\section{The Nagai-Honda Model}
The Honda-Nagai model, studied since the 1980s by the researchers H. Honda and T. Nagai, asserts that the the vertices in the mesh are acted upon by the force given by the gradient of the free energy function, which tends to want to minimize itself. The free energy function for each cell in this model is given by:
\begin{enumerate}
\item The deformation energy term \\ $U_D = \lambda(A - A_0)^2$ \\ where $A_0$ is a target area for a cell, and lambda is some positive constant.
\item The membrane surface energy \\ $U_S = \beta(C - C_0)^2$ \\ where $C$ is the cell perimiter, and $C_0$ is a target perimeter.
\item The cell-cell adhesion energy \\ $U_A = \displaystyle\sum\limits_{j = 0}^{n - 1}\gamma_{j}d_{j}$ \\ where $n$ is the number of vertices in the cell, $\gamma$ is some constant for the boundary in question between one cell and another, and d is the distance between one vertex and the next in a counter clockwise fashion. 
\end{enumerate}

As we have already mentioned, the \emph{gradient} of this free energy is what gives us the force acting on each vertex i.  As seen in \cite{ChasteMain}, this force on each vertex i is given by:

\begin{equation}
F_i = -\displaystyle\sum_{l\in N_i}(2\lambda(A_l - A_{0_l})\nabla_iA_l + 2\beta(C_l - C_{0_l})(\nabla_i d_{l, I_l-1}+\nabla_i d_{l, I_l}) + \gamma_{l, I_l-1}\nabla_i d_{l, I_l-1} + \gamma_{l, I_l}\nabla_i d_{l, I_l}
\end{equation} 
where l is the lth cell containing vertex $i$, given a counter clockwise orientation. $I_l$ is the local index of node $i$ in element $l$.

Furthemore, we have to specify the area and vertex-distance gradients. They are given by:

\begin{equation}
\nabla_i A_l = \frac12
\left(
\begin{array}{c}
y^l_{I+1} - y^l_{I-1}\\
x^l_{I-1} - x^l_{I+1}
\end{array}
\right)
\end{equation}
 where the superscrpits $l$ denote that x, y are in cell $l$. The subscripts are local indices in the cell $l$. The orientation is once again anti-clockwise.

\begin{equation}
\nabla_id_{l, j} = \frac1{d_{l, j}}
\left(
\begin{array}{c}
x_i - x_j\\
y_i - y_j
\end{array}
\right)
\end{equation}

where the subscript j denotes the neighboring vertex.

The other consideration of this model are T1 and T2 switches. These switches change the topology of the tissue in such a way that the cells never become concave, and edges never cross. A T1 swap is one that occurs when two vertices become too close to one another. Then the edge containing these vertices is flipped ninety degrees and two new vertices are introduced at a prespecified distance from each other. A T2 swap is other wise known as element removal, when a triangular element in the mesh shrinks below some minimal area, and the three vertices are replaced by one single vertex. 

The specified force, combined with these two elementary operations, constitute the Honda-Nagai model.

\section{Feedback Mechanisms and Proliferation}
This model is limited in that includes neither mechanical nor chemical feedback, and does not specify how cells proliferate. These are two extensions which can and have been made to the model by certain researchers. On the other hand, the dynamics of some successful models such as \cite{Morphogen} have the patterning of cells specified entirely by morphogens. In these models, chemicals called `morphogens' are what drive cells to proliferate, and the low concentrations of the morphogen on the fringe of the tissue explain why the tissue eventually ceases to grow as the tissue reaches a certain size.  Other models, such as those described in \cite{MechanicalFeedback} are based upon the vertex dynamics models such as the Honda-Nagai model, but include additional mechanical feedback terms which limit the growth of cells. 

\section{Additional Curiosities}
The vast majority of models assume that exactly three cells meet at any  vertex, except vertices on the boundary. In the language of topology, one would  say that all interior vertices in the tissue have degree three. Empirical observations show, however, that more more than three cells can meet at any junction under certain circumstances \cite{Vertex Models}. This means that models ought to be extended to include the formation of these `rosettes'.
